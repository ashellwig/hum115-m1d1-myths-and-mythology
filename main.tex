\documentclass[12pt]{article}

  \usepackage[english]{babel}
  \usepackage{hyperref}
  \usepackage{fancyhdr}
  \usepackage[dvipsnames]{xcolor}
  \usepackage{listings}
  \usepackage{parcolumns}
  \usepackage{algorithm}
  \usepackage{algorithmicx}
  \usepackage{algpseudocode}
  \usepackage{enumitem}
  \usepackage{geometry}
  \usepackage{soul}
  \usepackage{graphicx}
  \usepackage{enumitem}
  \usepackage{csquotes}
  \usepackage{bookmark}
  \usepackage{mdframed}
  \usepackage{mathtools}
  \usepackage{amsmath}
  \usepackage{amsthm}
  \usepackage[toc]{appendix}
  \usepackage[
    backend=biber,%
    style=apa%
  ]{biblatex}

  % Bibliography Setup
  \addbibresource{main.bib}
  \newcommand{\CiteSection}[2]{%
    (\autocite{#1}, ~\S {#1})
  }

  % Theorem Environments
  \theoremstyle{definition}
  \newtheorem*{defn*}{Definition}
  \theoremstyle{plain}
  \newtheorem*{equ*}{Equation}

  % Definitions for Algorithmic Environments
  \algdef{SE}[VARIABLES]{GVariables}{EndGVariables}
    {\algorithmicvariables}
    {\algorithmicend\ \algorithmicvariables}
  \algnewcommand{\algorithmicvariables}{\textbf{global variables}}
  
  \algdef{SE}[VARIABLES]{LVariables}{EndLVariables}
    {\algorithmiclvariables}
    {\algorithmicend\ \algorithmiclvariables}
  \algnewcommand{\algorithmiclvariables}{\textit{local variables}}

  \renewcommand{\algorithmicrequire}{\textbf{Input:}}
  \renewcommand{\algorithmicensure}{\textbf{Output:}}
  \renewcommand\thealgorithm{}

  % Settings for math-mode
  \makeatletter
  \def\mathcolor#1#{\@mathcolor{#1}}
  \def\@mathcolor#1#2#3{%
    \protect\leavevmode
    \begingroup
      \color#1{#2}#3%
    \endgroup
  }
  \makeatother


  % Image Directory
  \graphicspath{ {screenshots/} }
  % Hyperlink Setup
  \hypersetup{
    colorlinks = true,
    urlcolor = blue,
    linkcolor = blue,
    citecolor = blue
  }
  % Page and Text Layout
  \pagestyle{fancy}
  \geometry{%
    a4paper,%
    top=15pt,%
    bottom=1in,%
    left=1in,%
    right=1in%
  }
  \setlength{\headheight}{15pt}

  \newenvironment{ldefinitions}
    {\left.\begin{aligned}}
    {\end{aligned}\right\rbrace}

  \title{%
    Module 1 Discussion 1%
    \large{Myths and Mythology}
  }
  \author{Ashton Hellwig}
  \date{\today}
  \rhead{CSC160 Concept Discussion}

\begin{document}
  \maketitle
  \tableofcontents
  \lstlistoflistings
  \newpage


  \part{Initial Post}

    \section{Discussion Prompt}
      \begin{mdframed}
        \subsection{Overview}
          In this discussion we will examine what your belief of what a myth is
          and what purpose(s) it has.  You will need to read the Preface and
          Introduction to our textbook, (pgs. xi-xxi,) as a means to respond to
          this discussion.

        \subsection{Instructions}
          Compose a brief statement that clarifies your belief of what myth is
          and what purpose(s) it has. Give an example of some legend or tale
          from your experience and explain how the elements of that tale
          reinforced (or were intended to reinforce) particular customs or
          values. Please try to limit your contribution to an example that
          seems to have a purpose. Local ghost stories and/or odd strange events
          usually do not qualify. Compare your insights with at least two
          classmates.

          Be sure to proof-read your work (ALL rules of grammar, punctuation,
          spelling apply) and remember that I expect two well constructed
          paragraphs. I WILL deduct points for quick and shoddily written posts.
      \end{mdframed}

    \section{Response}
      \subsection{Placeholder}


  \newpage
  \part{Responses}

    \section{Response 1}
      \begin{quote}
        Reply to \textbf{} (\textit{Post ID:})
      \end{quote}
      Placeholder

    \section{Response 2}
      \begin{quote}
        Reply to \textbf{} (\textit{Post ID: }) 
      \end{quote}
      Placeholder

  % Bibliography
  \newpage
  \nocite{textbook}
  \printbibliography[
    heading=bibintoc,
    title={Bibliography}
  ]
\end{document}
