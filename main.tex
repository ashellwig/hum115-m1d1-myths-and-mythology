\documentclass[12pt]{article}

  \usepackage[english]{babel}
  \usepackage{hyperref}
  \usepackage{fancyhdr}
  \usepackage[dvipsnames]{xcolor}
  \usepackage{listings}
  \usepackage{parcolumns}
  \usepackage{algorithm}
  \usepackage{algorithmicx}
  \usepackage{algpseudocode}
  \usepackage{enumitem}
  \usepackage{geometry}
  \usepackage{soul}
  \usepackage{graphicx}
  \usepackage{enumitem}
  \usepackage{csquotes}
  \usepackage{bookmark}
  \usepackage{mdframed}
  \usepackage{mathtools}
  \usepackage{amsmath}
  \usepackage{amsthm}
  \usepackage[toc]{appendix}
  \usepackage[
    backend=biber,%
    style=apa%
  ]{biblatex}

  % Bibliography Setup
  \addbibresource{main.bib}
  \newcommand{\CiteSection}[2]{%
    (\autocite{#1}, ~\S {#1})
  }

  % Theorem Environments
  \theoremstyle{definition}
  \newtheorem*{defn*}{Definition}
  \theoremstyle{plain}
  \newtheorem*{equ*}{Equation}

  % Definitions for Algorithmic Environments
  \algdef{SE}[VARIABLES]{GVariables}{EndGVariables}
    {\algorithmicvariables}
    {\algorithmicend\ \algorithmicvariables}
  \algnewcommand{\algorithmicvariables}{\textbf{global variables}}
  
  \algdef{SE}[VARIABLES]{LVariables}{EndLVariables}
    {\algorithmiclvariables}
    {\algorithmicend\ \algorithmiclvariables}
  \algnewcommand{\algorithmiclvariables}{\textit{local variables}}

  \renewcommand{\algorithmicrequire}{\textbf{Input:}}
  \renewcommand{\algorithmicensure}{\textbf{Output:}}
  \renewcommand\thealgorithm{}

  % Settings for math-mode
  \makeatletter
  \def\mathcolor#1#{\@mathcolor{#1}}
  \def\@mathcolor#1#2#3{%
    \protect\leavevmode
    \begingroup
      \color#1{#2}#3%
    \endgroup
  }
  \makeatother


  % Image Directory
  \graphicspath{ {screenshots/} }
  % Hyperlink Setup
  \hypersetup{
    colorlinks = true,
    urlcolor = blue,
    linkcolor = blue,
    citecolor = blue
  }
  % Page and Text Layout
  \pagestyle{fancy}
  \geometry{%
    a4paper,%
    top=15pt,%
    bottom=1in,%
    left=1in,%
    right=1in%
  }
  \setlength{\headheight}{15pt}

  \newenvironment{ldefinitions}
    {\left.\begin{aligned}}
    {\end{aligned}\right\rbrace}

  \title{%
    Module 1 Discussion 1%
    \large{Myths and Mythology}
  }
  \author{Ashton Hellwig}
  \date{\today}
  \rhead{HUM115 Module 1 Discussion 1}

\begin{document}
  \maketitle
  \tableofcontents
  \lstlistoflistings
  \newpage


  \part{Initial Post}

    \section{Discussion Prompt}
      \begin{mdframed}
        \subsection{Overview}
          In this discussion we will examine what your belief of what a myth is
          and what purpose(s) it has.  You will need to read the Preface and
          Introduction to our textbook, (pgs. xi-xxi,) as a means to respond to
          this discussion.

        \subsection{Instructions}
          Compose a brief statement that clarifies your belief of what myth is
          and what purpose(s) it has. Give an example of some legend or tale
          from your experience and explain how the elements of that tale
          reinforced (or were intended to reinforce) particular customs or
          values. Please try to limit your contribution to an example that
          seems to have a purpose. Local ghost stories and/or odd strange events
          usually do not qualify. Compare your insights with at least two
          classmates.

          Be sure to proof-read your work (ALL rules of grammar, punctuation,
          spelling apply) and remember that I expect two well constructed
          paragraphs. I WILL deduct points for quick and shoddily written posts.
      \end{mdframed}

    \section{Response}
      \subsection{Placeholder}


  \newpage
  \part{Responses}

    \section{Response 1}
      \begin{mdframed}
        Reply to \textbf{Gabe Harris} (\textit{Post ID: 38558677})
      \end{mdframed}
      \begin{quote}
          I believe that a myth is a story, but not quite. What makes a myth different from a
            story, is that it is more than just a story. A myth is a story that has an intended
            purpose behind it. This purpose can be to explain why there’s a drought, such as the
            gods being unhappy, or to create a role model for what a citizen should be and someone
            to look up to, a hero. Myths are powerful, and can change peoples lives and change the
            course of history. This is not only true for the past, but even for the present.
            Personally, I have my own myth, though I’m not fully sure if it counts as one. When I
            was a kid, I was given these coins that had an angel on either side. I was told that as
            long as I had one on me, an angel would be watching over me. I did not know the validity
            of it, but I always kept one with me anyways, either in a wallet or my backpack. When I
            had my accident, I was lucky in so many ways. The car was destroyed, and I’d been flung
            into the back seat and would probably have died. A passing police officer was able to
            start CPR and saved my life, but that moment has always stuck with me. An angel was
            watching over me, and I’m convinced that it may have been the coin. So from then on, I
            always make sure to have the angel coin with me. Always. It’s not a crazy popular myth,
            but it’s my personal example of one.

          To me, a myth is a story with a purpose. Of course, my belief is not the same belief that
            everyone has. For some, myths are truth and there’s no way they could ever be made up.
            For others, myth’s are a way to explain the mind of humans, and to others it could be
            just an entertaining story! The reasons for myths vary, and there’s no one answer to
            why they are so powerful, or so prominent in old and new society. However, to me, a
            myth is a story. It’s a story with a purpose, either to explain the human mind, or why
            there is a drought destroying their lands. I believe humans need stories, as
            explanations for this crazy world of ours. So people create these stories, based on
            true life, in order to explain the world. These stories are then passed around from
            person to person, and become a myth. A myth, which explains this crazy world of ours.
      \end{quote}
      Placeholder.

    \section{Response 2}
      \begin{quote}
        Reply to \textbf{} (\textit{Post ID: }) 
      \end{quote}
      Placeholder.

  % Bibliography
  \newpage
  \nocite{textbook}
  \printbibliography[
    heading=bibintoc,
    title={Bibliography}
  ]
\end{document}
